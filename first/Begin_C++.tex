\documentclass[10pt,a4paper,fleqn]{ctexart}

\usepackage{amsmath}
\usepackage{graphicx}
\usepackage{wrapfig}
\usepackage{cite}
\usepackage{xltxtra}
\usepackage{float}
\usepackage{CJK}
\usepackage{xcolor}
\usepackage{hyperref}

\title{开始学习C++}
\author{唐浩 \\ 信息与计算科学 3200102118}

\begin{document}

\maketitle
\tableofcontents
\addcontentsline{toc}{section}{进入C++}
\addcontentsline{toc}{section}{C++语句}

\section{进入C++}

先展示一个简要的C++程序如下:
\begin{verbatim}
// myfirst.cpp -- displays a message

#include<iostream>   //a preprocessor directive
using namespace std;  //make definitions visible

int main(int argc, char *argv[])  //function header
{
    cout << "Come up and C++ me some time." << endl;  //message
    cout << "You won't reqret it!" << endl;
    return 0;  //terminate main()
}
\end{verbatim}

本人使用的是 {\bf linux} 系统 ({\bf ubuntu 20.04}), 采用的是 {\bf g++} 编译器,故输入命令: `{\bf g++ -o test myfirst.cpp}', 编译器运行之后将生成一个可执行文件 `test', 继续在终端 (emacs shell) 输入 `{\bf ./test}', 即可得到输出数据 :

Come up and C++ me some time.

You won't reqret it!

\subsection{main()函数}

每个C++程序都必须有一个 {\bf main()} 函数,其基本结构为:

\begin{verbatim}
int main(int argc, char*argv[])
{
  statements
  return 0;
}
\end{verbatim}

在运行 C++ 程序时,通常都是从 main() 函数开始执行,若无 main(), 则程序不完整,编译器将报错。

\subsection{C++注释}
C++ 的注释以双斜杠 (//) 打头。一个优秀的程序是不能没有注释的,注释将给读者带来一个良好的说明体验。注释是对读者的注释,编译器不会识别注释。

C++生成的注释从 // 开始,到行尾结束,可以单独位于一行,也可以和代码位于同一行。

\subsection{C++ 预处理器和 iostream 文件}
若程序要使用C++输入或输出工具,则需要提供这样的两行代码:

\begin{verbatim}
#include<iostream>
using namespace std;  //如果不使用,则在cout, endl等前需要加上 std::
\end{verbatim}

\#include<iostream> 将导致\textcolor{cyan}{预处理器}将 iostream 文件的内容添加到程序中。这是一种典型的预处理操作:在源代码被编译之前,替换或添加文本。iostream 中的 io 指的是输入和输出。C++的输入/输出方案涉及 iostream 文件中的多个定义。为了使用 cout 来显示消息,程序需要这些定义。\#include 编译指令会使 iostream 文件的内容随源代码文件的内容一起发给编译器。

\textcolor{red}{注意:使用 cin 和 cout 进行输入和输出的程序必须包含文件 iostream。}

\subsection{头文件名}
像 iostream 这样的文件叫做包含文件(include file)——由于它们被包含在其他文件中; 也叫\textcolor{-red}{头文件(header file)}——由于它们被包含在文件起始处。C++编译器自带了很多头文件,每个头文件都支持一组特定的工具。C++依旧可以使用C的头文件类型: xxx.h,但是C++的头文件其实是没有扩展名 h 的。例如,\textcolor{cyan}{C++版本的 math.h 为 cmath}(有些C头文件被转换成C++头文件,去掉了 .h, 加上了前缀 c)。

\subsection{名称空间}
using namespace xxx;

如果存在两个不同的函数,其名字的都是一样的,可以通过将其分别装入不同的名称空间,使编译器能准确了解用户所需的是哪个函数

\subsection{使用 cout 进行 C++ 输出}
cout << ``string" << endl;

cout将双引号中的内容打印出来,由于其不会自动移到下一行,故行末加上 endl, endl是一个特殊的 C++ 符号,表示一个重要的概念:重启一行。

C++ 还提供另外一种在输出中指示换行的旧式方法:C 语言符号 $\backslash n$:

cout << ``What's next?$\backslash n$";

\subsection{C++源代码格式化}
在C++中,分号标示了语句的结尾。因此在C++中,回车的作用就和空格或制表符相同。也就是说,在C++中,通常可以在能够使用回车的地方使用空格,反之亦然。

\section{C++语句}
\begin{verbatim}
//carrots.cpp

  #include <iostream>
  using namespace std;
  int main(int argc, char *argv[])
  {
    int carrots;

    carrots = 25;
    cout << ``I have ";
    cout << carrots;
    cout << `` carrots." << endl;
    carrots = carrots - 1;
    cout << ``Crunch, crunch. Now I have " << carrots << `` carrots." << endl;
    return 0;
  }
\end{verbatim}

\subsection{声明语句和变量}
在C++中,是使用声明语句来指出存储类型并提供位置标签的。例如:
int carrots;

它提供了两项信息: 需要的内存以及该内存单元的名称。int 表示整数。变量必须要声明。
\end{document}
